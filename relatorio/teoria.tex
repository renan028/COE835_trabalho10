\section{Backstepping - Formula��o te�rica com observador de ordem completa}

Neste trabalho, consideramos o sistema:

\begin{align}
\dot{x}_1 &= - a_2x_1 + x_2\\
\nonumber \dot{x}_2 &= -a_1x_1+x_3\\
\nonumber \dot{x}_3 &= -a_0x_1+k_pu \\
\nonumber y &= x_1
\label{eq:planta2}
\end{align}

onde os par�metros $a_2$, $a_1$, $a_0$ e $k_p$ s�o desconehcidos. Para esta
formula��o apenas a sa�da do sistema $y$ est� dispon�vel, portanto $x_2$ e $x_3$
n�o s�o conhecidos e devem ser estimados. Podemos reescrever o sistema
\ref{eq:planta2}:

\begin{align}
\nonumber \dot{x} &= Ax + F(y,u)^\intercal\theta \\
A &= 
\begin{bmatrix}
0 & 1 & 0\\
0 & 0 & 1\\
0 & 0 & 0
\end{bmatrix}, F(y,u)^\intercal = 
\nonumber \begin{bmatrix}
B(u) & \Phi(y)
\end{bmatrix}, \Phi(y) = 
\begin{bmatrix}
-y & 0 & 0\\
0 & -y & 0\\
0 & 0 & -y
\end{bmatrix}, B(u) = 
\begin{bmatrix}
0\\ 0\\u
\end{bmatrix}, \theta =
\begin{bmatrix}
k_p \\ a_2 \\
a_1 \\
a_0
\end{bmatrix} \\
\nonumber y &= e_1^\intercal x \\
\nonumber e_1 &= 
\begin{bmatrix}
1 \\ 0\\
0
\end{bmatrix} \\
\end{align}

Para estimar os estados, utilizamos os filtros abaixo:

\begin{align}
\label{eq:filtros2}
\dot{\xi} &= A_0\xi + ky \\
\nonumber \dot{\Omega}^\intercal &= A_0\Omega^\intercal + F^\intercal\\
\nonumber k &=
\begin{bmatrix}
k_1\\k_2\\k_3
\end{bmatrix}, A_0 = A - ke_1^\intercal =  
\begin{bmatrix}
-k_1 & 1 & 0\\-k_2 & 0 & 1\\-k_3 & 0 & 0
\end{bmatrix}
\end{align}

Os valores de $k$ devem ser escolhidos de forma que $A_0$ seja Hurwitz. E, dessa
forma, o estado estimado pode ser escrito como:

\begin{align}\label{eq:estimador}
\hat{x} = \xi + \Omega^\intercal\theta
\end{align}

Derivando a equa��o \ref{eq:estimador} e substituindo as equa��es dos
filtros \ref{eq:filtros2}, verifica-se que a din�mica do estimador � igual �
din�mica da planta \ref{eq:planta2}.

Por�m, $\Omega$ � uma matriz e opta-se pela redu��o das ordens dos
filtros. Observe que $\Omega^\intercal = \left[v_0 \quad | \quad \Xi\right]$ e,
pela equa��o ~\ref{eq:filtros2}, temos que:

\begin{align}
\dot{v}_0 &= A_0v_0 + e_3u \\
\dot{\Xi} &= A_0\Xi - Iy
\end{align}

Introduzem-se dois novos filtros, para substituir os filtros da
equa��o~\ref{eq:filtros2}:

\begin{align}
\dot{\lambda} &= A_0\lambda + e_3u \\
\dot{\eta} &= A_0\eta + e_3y
\end{align}

� f�cil verificar que, para esta planta de segunda ordem sem zeros ($m=0$), $v_0
= \lambda$. Para o caso geral, temos que:

\begin{align}
\dot{\lambda} &= A_0\lambda + e_3u \\
v_i &= A_0^i\lambda \quad (i=0,\ldots,m)
\end{align}

� poss�vel demonstrar que:
\begin{align}
\Xi &= -\left[A_0^2\eta \quad A_0\eta \quad \eta\right] \\
\xi &= -A_0^3\eta
\end{align}

Podemos reescrever a din�mica da sa�da $y$:

\begin{align}
\dot{y} &= k_pv_{0,2} + \xi_2 + \bar{\omega}^\intercal\theta + \epsilon_2 \\
\nonumber \bar{\omega}^\intercal &= 
\begin{bmatrix}
0 & (\Xi_2+\phi_1^\intercal)
\end{bmatrix}
\end{align}

Desta forma, o sistema \ref{eq:planta2} pode ser representado com os estados do
observador:
\begin{align}
\dot{y} &= k_pv_{0,2} + \xi_2 + \bar{\omega}^\intercal\theta + \epsilon_2\\
\nonumber \dot{v}_{0,2} &= v_{0,3} - k_2v_{0,1}\\
\nonumber \dot{v}_{0,3} &= - k_3v_{0,1} + u\\
\end{align}

O projeto backstepping agora segue como na se��o anterior. Primeiro, fazemos a
mudan�a de coordenadas em \textbf{z}:

\begin{align}
z_1 &= y - y_r \\
\nonumber z_2 &= v_{0,2} - \alpha_{1} - \hat{\rho}\dot{y_r} \\
\nonumber z_3 &= v_{0,3} - \alpha_{2} - \hat{\rho}\ddot{y_r} 
\end{align}

onde $\rho$ � estimativa de $\frac{1}{k_p}$. Os controles virtuais $\alpha_1$ e
$alpha_2$, a lei de controle $u$ e as leis de adapta��o $\dot{\theta}$ e
$\dot{\rho}$ s�o obtidas pelo m�todo de Lyapunov. A formula��o e desenvolvimento
dos passos da solu��o do problema s�o realizados nas notas de aula. Dessa forma,
neste relat�rio fazemos o resumo das equa��es:

\begin{align*}
\alpha_1 &= \hat{\rho}\bar{\alpha}_1 = \hat{\rho}(-c_1z_1 - d_1z_1 - \xi_2 -
\bar{\omega}^\intercal\hat{\theta})\\
%
\dot{\hat{\rho}} &= -\gamma
\text{sign}(k_p)\left[\dot{y}_r+\bar{\alpha}_1\right]z_1 \\
%
\tau_1 &= \left[\omega-\hat{\rho}(\dot{y}_r+\bar{\alpha}_1)e_1\right]z_1\\
%
\beta_2 &= k_2v_{0,1} + \frac{\partial \alpha_1}{\partial y}(\xi_2 +
\omega^\intercal \hat{\theta}) + \frac{\partial \alpha_1}{\partial
\eta}(A_0\eta + e_3 y) + \frac{\partial \alpha_1}{\partial y_r}\dot{y}_r +
\left(\dot{y}_r+\frac{\partial \alpha_1}{\partial
\hat{\rho}}\right)\dot{\hat{\rho}}\\
%
\tau_2 &= \tau_1 - \frac{\partial \alpha_1}{\partial y}\omega z_2 \\
%
\alpha_2 &= -c_2z_2 - \hat{k}_pz_1 + \beta_2 + \frac{\partial
\alpha_1}{\partial \hat{\theta}}\Gamma \tau_2 - d_2
\left(\frac{\partial \alpha_1}{\partial y}\right)^2z_2 \\
%
\beta_3 &= k_3v_{0,1} + \frac{\partial \alpha_2}{\partial y}(\xi_2 +
\omega^\intercal \hat{\theta}) + \frac{\partial \alpha_2}{\partial
\eta}(A_0\eta + e_3 y) + \frac{\partial \alpha_2}{\partial
\lambda_1}(-k_1\lambda_1+\lambda_2) + \frac{\partial
\alpha_2}{\partial\lambda_2}(-k_2\lambda_2+\lambda_3) +\\
%
&+ \frac{\partial
\alpha_2}{\partial y_r}\dot{y}_r +\frac{\partial
\alpha_2}{\partial \dot{y}_r}\ddot{y}_r +
\left(\ddot{y}_r+\frac{\partial\alpha_2}{\partial\hat{\rho}}\right)\dot{\hat{\rho}}\\
%
\tau_3 &= \tau_2 - \frac{\partial \alpha_2}{\partial y}\omega z_3\\
%
\dot{\hat{\theta}} &= \Gamma\tau_3 \\
%
u &= -c_3z_3 + \beta_3 + \hat{\rho}\dddot{y}_r +
\frac{\partial\alpha_2}{\partial \hat{\theta}}\Gamma\tau_3 -
d_3\left(\frac{\partial\alpha_2}{\partial y}\right)^2z_3 - z_2
\frac{\partial\alpha_1}{\partial\hat{\theta}}\Gamma\frac{\partial\alpha_2}{\partial
y}\omega
\end{align*}

