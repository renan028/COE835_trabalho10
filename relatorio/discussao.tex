%---------------------------------------------------------------------
\section{Discuss�o}

A \textbf{simula��o \#1} mostra o comportamento do sistema para varia��es nas
condi��es iniciais. A rapidez da converg�ncia depende de qu�o pr�ximo os par�metros estimados est�o
dos par�metros reais. A simula��o mostrou um comportamento semelhante para ambos
os casos. Quando deslocamos o $y(0)$, na simula��o 1.2, os sistemas tamb�m
apresentam comportamento semelhante, pois a vari�vel de controle $u$ � alterada
e n�o apresenta satura��o, compensando a condi��o inicial.

A \textbf{simula��o \#2} mostra o comportamento do sistema para varia��es no
ganho de adapta��o $\Gamma$. Podemos verificar converg�ncia mais r�pida quando o
$\Gamma$ � maior, por�m tamb�m observamos maiores oscila��es e picos de erro.

A \textbf{simula��o \#3} mostra o comportamento do sistema para varia��es na
planta. Escolhemnos plantas inst�veis e est�veis. Podemos observar que, em ambos
os casos, o sinal de controle foi capaz de corrigir o erro, sem muitas
dificuldades. O comportamento dos sistemas � semelhante. A altera��o do modelo,
na simula��o 3.2, mostrou que modelos mais complexos, com frequ�ncias mais altas
e maiores amplitudes, demoram mais para convergir.