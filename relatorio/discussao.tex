%---------------------------------------------------------------------
\section{Discuss�o}

Conforme aumentamos a ordem do Backstepping, as derivadas parciais crescem em
grande complexidade, o que requer grande trabalho manual para otimizar o
algoritmo. Al�m disso, o aumento do n�mero de multiplica��es e de novos termos
impede que a simula��o seja r�pida. Verificou-se tamb�m a necessidade da
diminui��o do ganho de adapta��o, se comparado com outros agoritmos como o MRAC,
devido �s multiplica��es e poss�vel overflow num�rico.

A \textbf{simula��o \#1} testou o sistema para condi��es iniciais n�o nulas
tanto dos par�metros $\theta$, quando do sistema $y(0)$. Observou-se a
converg�ncia do erro $e_0$ para zero. Os par�metros convergiram para valores
pr�ximos aos valores esperados.

A \textbf{simula��o \#2} testou o sistema para diferentes valores de
ganho de adapta��o: $\Gamma = 0.1$ e $\Gamma = 0.5$. Para o maior valor,
$\Gamma = 0.5$, o sistema convergiu mais rapidamente, com menor erro, por�m com
maiores oscila��es. Observou-se overflow num�rico para valores $\Gamma > 10$.

A \textbf{simula��o \#3.1} testou o algoritmo para diferentes sistemas,
modelos de tranfer�ncia: $Y/U = \frac{5}{s^3+3s^2+3s+1}$ e $Y/U =
\frac{5}{s^3-3s^2-3s+1}$. Para a planta inst�vel, o algoritmo leva muito tempo
para convergir e as oscila��es s�o maiores.

A \textbf{simula��o \#3.2} testou o algoritmo para diferentes entradas: $r_1 =
\text{sin}(t) + \text{sin}(3t)$ e $r_2 = \text{sin}(t) + 2\text{sin}(5t)$.
Apesar de a entrada de alta frequ�ncia exigir mais do sistema, mostrando maior
lentid�o para a converg�ncia, os par�metros convergiram mais rapidamente.

